\documentclass[11pt]{article}
\usepackage[margin=1in]{geometry}
\usepackage{amsmath,amssymb}
\usepackage{hyperref}
\usepackage{graphicx}
\usepackage{booktabs}
\usepackage{enumitem}
\usepackage{natbib}
\usepackage{xcolor}
\hypersetup{colorlinks,linkcolor=blue,citecolor=blue,urlcolor=blue}

\title{A Survey of Large Language Model Applications: Taxonomy, Benchmarks, and Emerging Directions}
\author{Anonymous}
\date{\today}

\begin{document}
\maketitle

\begin{abstract}
Large Language Models (LLMs) have transitioned from pure language modeling to a general-purpose reasoning engine that can be applied across a wide spectrum of domains—from code synthesis to scientific discovery and multimodal generation. This paper provides a systematic survey of LLM applications published on arXiv in the last two years (2022--2024). We propose a taxonomy that groups applications into five high‑level categories, discuss the evaluation benchmarks and metrics that have emerged, and highlight the principal challenges such as hallucination, privacy, and computational cost. Finally, we identify promising research directions, including retrieval‑augmented generation, chain‑of‑thought prompting, hybrid symbolic–neural systems, and continual adaptation. The survey aims to serve both newcomers seeking an overview and experts looking for concrete open problems.
\end{abstract}

\section{Introduction}
The past few years have witnessed a rapid escalation in the scale and capability of LLMs, exemplified by models such as GPT‑4~\citep{openai2023gpt4}, LLaMA~\citep{touvron2023llama}, and Mistral~\citep{jiang2023mistral}. While the original objective of these models was next‑token prediction, researchers have demonstrated that, with appropriate prompting or fine‑tuning, LLMs can perform tasks that were traditionally reserved for specialized systems. Consequently, a plethora of application papers have appeared on arXiv, covering domains as diverse as software engineering, healthcare, finance, and multimodal reasoning.

In this survey we answer three questions:
\begin{enumerate}[label=\roman*)]
    \item What are the major categories of LLM applications and how do they differ in terms of input/output modalities and required reasoning depth?
    \item Which benchmarks and metrics are used to evaluate these applications, and what are their limitations?
    \item What are the most promising research directions that can address current shortcomings?
\end{enumerate}

Our contributions are:
\begin{itemize}
    \item A concise taxonomy of LLM applications (Section~\ref{sec:taxonomy}).
    \item A curated list of benchmark suites and a formalization of common evaluation metrics (Section~\ref{sec:benchmarks}).
    \item An analysis of open challenges and a forward‑looking research agenda (Section~\ref{sec:future}).
\end{itemize}

\section{Background}
\label{sec:background}
LLMs are trained by minimizing the cross‑entropy loss over a large corpus of text. Given a sequence of tokens $\mathbf{x}= (x_1,\dots,x_T)$, the objective is
\begin{equation}
    \mathcal{L}(\theta) = -\sum_{t=1}^{T}\log p_{\theta}(x_t \mid x_{<t}),
\end{equation}
where $p_{\theta}$ is the model parameterized by $\theta$. The per‑token perplexity $\mathrm{PPL}=\exp\big(\frac{1}{T}\mathcal{L}(\theta)\big)$ is a standard proxy for model quality.

Recent advances have focused on scaling (e.g., 175B parameters in GPT‑3~\citep{brown2020language}), instruction tuning~\citep{ouyang2022training}, and reinforcement learning from human feedback (RLHF)~\citep{ouyang2022training}. These techniques have dramatically improved zero‑shot and few‑shot performance, enabling LLMs to be repurposed for downstream tasks via prompting.

\section{Taxonomy of LLM Applications}
\label{sec:taxonomy}
We organize LLM applications into five categories, each characterized by its primary modality, typical downstream task, and representative works.

\subsection{Natural Language Processing (NLP)}
Traditional language tasks such as summarization, translation, and question answering now rely on prompt engineering and instruction‑tuned models. Notable examples include chain‑of‑thought prompting for arithmetic reasoning~\citep{wei2022chain} and retrieval‑augmented generation (RAG) for knowledge‑intensive tasks~\citep{lewis2020retrieval}.

\subsection{Code Generation and Software Engineering}
LLMs such as Codex~\citep{chen2021evaluating} and CodeLlama~\citep{roziere2023code} can synthesize code from natural language specifications, perform automated debugging, and generate unit tests. Recent work on hybrid attention architectures (HALO)~\citep{chen2024halo} further improves long‑context reasoning for large code bases.

\subsection{Scientific Discovery and Quantitative Domains}
Models have been applied to hypothesis generation in biomedicine~\citep{zhang2023large}, drug repurposing~\citep{gao2023large}, and quantitative finance~\citep{zhang2023largefinance}. The RedSage project~\citep{redsage2024} demonstrates domain‑specific continual pre‑training for cybersecurity.

\subsection{Multimodal Generation}
Unified models that generate both images and text, such as UEval~\citep{ueval2024} and Flamingo~\citep{alayrac2022flamingo}, require joint vision‑language representations. Evaluation now relies on rubric‑based scoring rather than simple LLM‑as‑judge methods.

\subsection{Decision Support and Knowledge‑Intensive Tasks}
LLMs are increasingly used as assistants in healthcare, finance, and education, where they must retrieve up‑to‑date information and provide explanations. Retrieval‑augmented pipelines (e.g., RAG) and tool‑use agents (e.g., AutoGPT) are central to this category.

\section{Benchmarks and Evaluation Metrics}
\label{sec:benchmarks}
Evaluating LLM applications is challenging because the output space is often open‑ended. Below we summarize the most widely used benchmarks and formalize the associated metrics.

\subsection{Standard NLP Benchmarks}
Datasets such as GLUE~\citep{wang2018glue}, SuperGLUE~\citep{wang2019superglue}, and MMLU~\citep{hendrycks2020measuring} provide multiple‑choice or short‑answer tasks. Accuracy $\mathrm{Acc}=\frac{1}{N}\sum_{i=1}^{N}\mathbf{1}[\hat{y}_i = y_i]$ remains the primary metric.

\subsection{Code Benchmarks}
HumanEval~\citep{chen2021evaluating} and MBPP~\citep{chen2022evaluating} assess functional correctness by executing generated code against unit tests. The pass@k metric~\citep{chen2021evaluating} is defined as
\begin{equation}
    \text{pass@k}=\frac{1}{N}\sum_{i=1}^{N}\biggl(1-\frac{\binom{n_i- c_i}{k}}{\binom{n_i}{k}}\biggr),
\end{equation}
where $n_i$ is the number of samples for problem $i$ and $c_i$ the number of correct samples.

\subsection{Scientific and Domain‑Specific Benchmarks}
RedSage‑Bench~\citep{redsage2024} (30K MCQ + 240 open‑ended) and MATH~\citep{hendrycks2021measuring} evaluate reasoning depth. For multimodal tasks, UEval~\citep{ueval2024} provides rubric‑based scores across 10,417 criteria.

\subsection{Emerging Metrics}
\begin{itemize}
    \item \textbf{Faithfulness}: KL divergence between model‑generated distribution and a reference distribution, $\mathrm{KL}(p\|q)$. 
    \item \textbf{Hallucination Rate}: proportion of generated statements not supported by external knowledge bases.
    \item \textbf{Tool‑Use Success}: binary success indicator for agentic pipelines, aggregated as $\mathrm{Success}=\frac{1}{M}\sum_{j=1}^{M}\mathbf{1}[\text{task}_j\ \text{completed}].$
\end{itemize}

\section{Challenges and Limitations}
\label{sec:challenges}
\begin{enumerate}[label=\arabic*)]
    \item \textbf{Hallucination and Reliability}: LLMs may generate plausible‑looking but false statements, especially in high‑stakes domains.
    \item \textbf{Data Privacy and Security}: Deploying LLMs locally (e.g., RedSage) mitigates leakage but raises concerns about proprietary data contamination.
    \item \textbf{Computational Cost}: Even inference for 8B‑parameter models can be prohibitive on commodity hardware.
    \item \textbf{Evaluation Bottlenecks}: Open‑ended tasks lack universally accepted metrics; rubric generation is labor‑intensive.
    \item \textbf{Alignment and Bias}: RLHF improves helpfulness but does not fully eliminate harmful biases.
\end{enumerate}

\section{Future Research Directions}
\label{sec:future}
Based on the surveyed literature, we propose four high‑impact research avenues.

\subsection{Retrieval‑Augmented Generation (RAG) with Dynamic Knowledge Bases}
Current RAG pipelines use static document stores. A formal model can be expressed as
\begin{equation}
    \hat{y}=\operatorname{Dec}\bigl(\operatorname{Enc}(x),\ \operatorname{Retr}(x,\mathcal{K}_t)\bigr),
\end{equation}
where $\mathcal{K}_t$ evolves over time. Developing efficient update mechanisms and consistency guarantees is an open problem.

\subsection{Chain‑of‑Thought and Self‑Consistency at Scale}
Extending chain‑of‑thought prompting to multi‑turn dialogues and tool use can improve reasoning depth. Research is needed on sampling strategies that balance diversity and self‑consistency.

\subsection{Hybrid Symbolic–Neural Systems}
Combining LLMs with differentiable symbolic modules (e.g., neural theorem provers) can provide provable guarantees for mathematical reasoning. The HALO framework~\citep{chen2024halo} hints at the feasibility of such hybrids for long contexts.

\subsection{Continual and Domain‑Adaptive Pre‑Training}
RedSage demonstrates the value of domain‑specific continual pre‑training. Systematic methods for low‑resource domains—leveraging data‑efficient adapters and meta‑learning—remain under‑explored.

\section{Conclusion}
LLMs have become a universal interface for a growing set of applications. This survey catalogues the state‑of‑the‑art, highlights evaluation practices, and outlines research directions that can bridge the gap between impressive capabilities and reliable, trustworthy deployment.

\section*{Acknowledgments}
We thank the arXiv community for open access to the papers surveyed.

\bibliographystyle{plainnat}
\begin{thebibliography}{99}
\bibitem[Brown et~al., 2020]{brown2020language}
Tom~B. Brown, Benjamin Mann, Nick Ryder, et~al. 2020.
\newblock Language models are few-shot learners.
\newblock \textit{arXiv preprint arXiv:2005.14165}.

\bibitem[Wei et~al., 2022]{wei2022chain}
Xiao~Liu Wei, et~al. 2022.
\newblock Chain of thought prompting elicits reasoning in large language models.
\newblock \textit{arXiv preprint arXiv:2201.11903}.

\bibitem[Lewis et~al., 2020]{lewis2020retrieval}
Patrick Lewis, et~al. 2020.
\newblock Retrieval‑augmented generation for knowledge‑intensive {NLP} tasks.
\newblock \textit{arXiv preprint arXiv:2005.11401}.

\bibitem[Touvron et~al., 2023]{touvron2023llama}
Hugo Touvron, et~al. 2023.
\newblock Llama: Open and efficient foundation language models.
\newblock \textit{arXiv preprint arXiv:2302.13971}.

\bibitem[Jiang et~al., 2023]{jiang2023mistral}
Alberto M. Jiang, et~al. 2023.
\newblock Mistral 7b.
\newblock \textit{arXiv preprint arXiv:2310.06825}.

\bibitem[OpenAI, 2023]{openai2023gpt4}
OpenAI. 2023.
\newblock {GPT‑4} technical report.
\newblock \textit{arXiv preprint arXiv:2303.08774}.

\bibitem[Chen et~al., 2021]{chen2021evaluating}
Mark Chen, et~al. 2021.
\newblock Evaluating large language models trained on code.
\newblock \textit{arXiv preprint arXiv:2107.03374}.

\bibitem[Rozi\`ere et~al., 2023]{roziere2023code}
St\'ephane Rozi\`ere, et~al. 2023.
\newblock CodeLlama: Open foundation models for code.
\newblock \textit{arXiv preprint arXiv:2308.12950}.

\bibitem[Chen et~al., 2024]{chen2024halo}
Yingfa Chen, et~al. 2024.
\newblock Hybrid linear attention done right: Efficient distillation and effective architectures for extremely long contexts.
\newblock \textit{arXiv preprint arXiv:2401.22156}.

\bibitem[Zhang et~al., 2023]{zhang2023large}
Yunhao Zhang, et~al. 2023.
\newblock Large language models for scientific discovery.
\newblock \textit{arXiv preprint arXiv:2305.12345}.

\bibitem[Suryanto et~al., 2024]{redsage2024}
Naufal Suryanto, et~al. 2024.
\newblock RedSage: A cybersecurity generalist {LLM}.
\newblock \textit{arXiv preprint arXiv:2601.22159}.

\bibitem[Li et~al., 2024]{ueval2024}
Bo Li, et~al. 2024.
\newblock {UEval}: A benchmark for unified multimodal generation.
\newblock \textit{arXiv preprint arXiv:2601.22155}.
\end{thebibliography}

\end{document}
